\begin{multicols}{2}
    \subsubsection*{Python}
    
    \begin{lstlisting}[language=Python]
        import random
        import itertools
    
        #randint: retorna um numero aleatorio x tq. a <= x <= b
        lista = [random.randint(1,100) for i in range(101)]
    
        #shuffle: embaralha uma sequencia
        random.shuffle(lista)
    
        #sample: retorna uma lista de k elementos unicos escolhidos de uma sequencia
        amostra = random.sample(lista, k = 10)
    
        lista2 = [1,2,3,4,5]
        #permutations: iterable que retorna permutacoes de tamanho r
        permutacoes = [perm for perm in itertools.permutations(lista2, 2)]
        
        #combinations: iterable que retorna combinacoes de tamanho r (ordenado)
        #combinations_with_replacement: combinations() com elementos repetidos
        combinacoes = [comb for comb in itertools.combinations(lista2, 2)]
    \end{lstlisting}
    
    \subsubsection*{C++}
    
    \begin{lstlisting}[language=c++]
        mt19937 rng(chrono::steady_clock::now().time_since_epoch().count()); // mt19937_64
        uniform_int_distribution<int> distribution(1,n);
    
        num = distribution(rng); // num no range [1, n]
        shuffle(vec.begin(), vec.end(), rng); // shuffle
    
        // permutacoes 
        do {
            // codigo
        } while(next_permutation(vec.begin(), vec.end()))
    
        using ull = unsigned long long;
        ull mix(ull o){
            o+=0x9e3779b97f4a7c15;
            o=(o^(o>>30))*0xbf58476d1ce4e5b9;
            o=(o^(o>>27))*0x94d049bb133111eb;
            return o^(o>>31);
        }
    
        ull hash(pii a) {return mix(a.first ^ mix(a.second));}
    \end{lstlisting}
    
\end{multicols}