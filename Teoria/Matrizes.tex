\subsubsection{Determinante} 
A determinante $det(A)$ de uma matriz $A$ é definida se $A$ é uma matriz quadrada. Se $A$ é de tamanho $1\times 1$, então $det(A)$ = $A_{11}$. A determinante de matrizes maiores é calculada recursivamente usando a fórmula:
$$det(A) = \sum_{j=1}^{m}A_{1,j}C_{1,j},$$
onde $C_{i,j}$ é o \textbf{cofator} de $A$ em $i,j$. O cofator é calculado usando a fórmula:
$$C_{i,j} = (-1)^{i+j}det(M_{i,j}),$$
onde $M_{i,j}$ é obtido ao remover a linha $i$ e a coluna $j$ de $A$.

A determinante de $A$ indica se existe uma \textbf{matriz inversa} $A^{-1}$ tal que $AA^{-1} = I$, onde $I$ é uma matriz identidade. $A^{-1}$ existe somente quando $det(A) \neq 0$, e pode ser calculada usando a fórmula:
$$A^{-1}_{i,j} = \frac{C_{i,j}}{det(A)}.$$

\subsubsection{Aplicações}
\paragraph{Contando caminhos.} Quando $V$ é a matriz de adjacência de um grafo sem peso, a matriz $V^n$ contém o número de caminhos de $n$ arestas entres os vértices do grafo.

\paragraph{Menores caminhos.} Usando uma ideia similar para grafos com peso, podemos calcular para cada par de vértices a distância mínima para um caminho entre eles no qual contém exatos $n$ arestas. Construímos a matriz de adjacência onde $\infty$ significa que uma aresta não existe, e outros valores correspondem ao peso da aresta. Utilizamos a fórmula
$$AB_{i,j} = \min_{k=1}^{n} A_{i,k}+B_{k,j}$$
na multiplicação de matriz. Após essa modificação, as potências da matriz correspondem aos menores caminhos no grafo.

\paragraph{Teorema de Kirchhoff.} Para calcular o número de árvores geradoras de um grafo, construímos uma matrix laplaciana $L$, onde $L_{i,i}$ é o grau do vértice $i$ e $L_{i,j} = -1$ se há uma aresta entre os vértices $i$ e $j$, caso contrário $L_{i,j} = 0$. O número de árvores geradoras é igual à determinante da matriz que é obtida ao removermos qualquer linha e qualquer coluna de $L$.