\subsubsection{Caminhos}
\subsubsection*{Caminho de Euler} 
Um caminho de Euler em um grafo é o caminho que visita cada aresta exatamente uma vez. Um ciclo de Euler, ou Tour de Euler, em um grafo é um ciclo que usa cada aresta exatamente uma vez.

\paragraph{Teorema:} Um grafo conectado tem um ciclo de Euler se, e somente se, cada vértice possui grau par.

\subsubsection*{Caminho Hamiltoniano} 
Um caminho Hamiltoniano em um grafo é o caminho que visita cada vértice exatamente uma vez. Um ciclo Hamiltoniano em um grafo é um ciclo que visita cada vértice exatamente uma vez.

\paragraph{Teoremas:} \empty
\begin{itemize}
    \item Teorema de Dirac: Um grafo simples com $n$ vértices $(n\ge 3)$ é Hamiltoniano se cada vértice tem grau $ \ge \frac{n}{2}$.
    \item Teorema de Ore: Um grafo simples com $n$ vértices $(n\ge 3)$ é Hamiltoniano se, para cada par de vértices não-adjacentes, a soma de seus graus é $\ge n$.
    \item Ghouila-Houiri: Um grafo direcionado simples fortemente conexo com $n$ vértices é Hamiltoniano se cada vértice tem um grau $\ge n$.
    \item Meyniel: Um grafo direcionado simples fortemente conexo com $n$ vértices é Hamiltoniano se a soma dos graus de cada par de vértices não-adjacentes é $\ge 2n-1$.
\end{itemize}